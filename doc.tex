\documentclass[a4paper, fontset=adobe , punct=zh_CN/kaiming]{ctexart}
\usepackage{luatexja-adjust}

\def\+{\insertxkanjiskip}

\usepackage[textwidth=36em , centering]{geometry}

\usepackage{hyperref}
\hypersetup{colorlinks=true}

\title{\textbf{\textsf{ChineseJFM}文件}\hbox{}\thanks{\url{https://github.com/tanukihee/ChineseJFM}}}
\author{ListLee}
\date{\zhtoday \qquad v1.0.1}

\begin{document}
\maketitle

\section{概述}
\textsf{ChineseJFM}文件是为中文排版编写的luatexja-jfm文件,提供全角、半角、开明三种风格,适用于简繁中文及日文字体的横直排。

虽然\textsf{luatexja}自带了一套全角、半角、开明的JFM文件,但在调整上缺少灵活性,同时也没有使用\+\verb|priority|特性。\textsf{ChineseJFM}文件中同时增加了\+\verb|priority|的设置,配合\textsf{luatexja-adjust}包,可以进行有优先顺序的标点挤压。\label{sec:pr}

\section{使用方法}
\textsf{luatexja}在20200919版本后更新了JFM features特性,基本语法为
\begin{verbatim}
    jfm = <JFM name>[/{<JFM features>}]
\end{verbatim}

\textsf{ChineseJFM}提供了\textsf{zh\_CN}、\textsf{zh\_TW}与\textsf{ja\_JP}三个文件,分别适用于简中字体(标点偏靠)、繁中字体(标点中置)与日文字体(冒号、分号中置,其他偏靠);在3个JFM文件下,提供了\+\verb|quanjiao|、\verb|banjiao|与\+\verb|kaiming|三个基本特性。

在使用\CTeX{}宏集或文档类时,只需要使用
\begin{verbatim}
    \documentclass[punct = zh_CN/quanjiao]{ctexart}
\end{verbatim}
或
\begin{verbatim}
    \usepackage[punct = zh_CN/kaiming]{ctex}
\end{verbatim}
即可。

在使用\textsf{luatexja}或\textsf{ltj}文档类时,需要在调用\textsf{luatexja-fontspec}宏包后,以\begin{verbatim}
    YokoFeatures = {JFM = {<JFM name>}}
\end{verbatim}
或
\begin{verbatim}
    TateFeatures = {JFM = {<JFM name>}}
\end{verbatim}
的形式调用。

横排时,使用
\begin{verbatim}
    YokoFeatures = {JFM = {zh_CN/quanjiao}}
\end{verbatim}
直排时,使用
\begin{verbatim}
    TateFeatures = {JFM = {zh_CN/{quanjiao , vert}}}
\end{verbatim}
{\color{red}注意:在直排时,\emph{必须}同时使用\+\verb|vert|特性。}

\section{特性一览}
除上文介绍的\+\verb|quanjiao|、\verb|banjiao|、\verb|kaiming|与\+\verb|vert|特性外,\textsf{ChineseJFM}还提供了以下特性。

\begin{description}
    \item[\texttt{hwcl}] “half-width colon”,“半宽冒号”特性。适用于简中直排时,冒号分号只占半宽的字体(如华文宋体等),仅\textsf{zh\_CN}拥有。
    \item[\texttt{prop}] “proportional”,“比例宽度”特性。适用于比例宽度的日文字体,比之于\textsf{luatexja}自带的\textsf{jfm-prop.lua}与\textsf{jfm-propv.lua},\verb|prop|特性将此两者整合的同时,也提供了对标点间距的调整功能,仅\textsf{ja\_JP}拥有。
\end{description}

\section{挤压顺序}
如第\ref{sec:pr}节所述,\textsf{ChineseJFM}文件中配有\+\verb|priority|的设置,配合\textsf{luatexja-adjust}包,可以进行有优先顺序的标点挤压。\textsf{ChineseJFM}沿袭传统铅字排版与现代DTP软件(如InDesign等)的习惯,认为标点只占半宽,全宽标点是半宽标点加上半宽铅空的结果。所谓的“标点挤压”更应说成是“标点间距调整”,调整顺序如下。
\begin{itemize}
    \item 最先给句号、问号、叹号插空;
    \item 其次给顿号、逗号、冒号、分号插空;
    \item 最后调整引号、括号前后与间隔号两边的空格;
    \item 如果进行上述调整后,仍无法达到行长要求,最后才会进行字间字距调整。
\end{itemize}

根据标点位置,可以将标点分为偏靠标点、中置标点与全宽标点。对于偏靠标点,插空指在半宽标点后插入一个半宽铅空,对于中置标点,插空指在半宽标点前后各插入一个1/4宽铅空。全宽标点无空白可调整的,不作调整。

\nocite{*}
\begin{thebibliography}{9}
\bibitem{孔雀计划} 刘庆(Eric Q L).\textit{孔雀计划:中文字体排印的思路} [EB/OL].\\
\raggedleft{\url{https://thetype.com/kongque/}}
\end{thebibliography}

\end{document}