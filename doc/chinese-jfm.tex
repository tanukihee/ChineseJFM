% The documentation of chinese-jfm
% v2.0.0
% Copyright (c) 2020 -- 2023 ListLee.

\documentclass[a4paper , zihao=-4 , punct=zh_CN/kaiming]{ctexart}
\usepackage{luatexja-adjust}
% \usepackage{lua-visual-debug}
\protect\def\+{\insertxkanjiskip}

\usepackage[textwidth=34em , centering]{geometry}

\usepackage{hyperref}
\hypersetup{colorlinks=true}

\title{\textbf{\textsf{Chinese JFM}}\thanks{\url{https://github.com/tanukihee/ChineseJFM}}}
\author{ListLee}
\date{\zhtoday \qquad v2.0.0}

\begin{document}
\maketitle

\section{概述}
\textsf{Chinese JFM}是为中文排版编写的luatexja-jfm文件,提供全角、半角、开明三种风格,适用于简繁中文及日文字体的横直排。

虽然\textsf{luatexja}自带了一套全角、半角、开明的JFM文件,但在调整上缺少灵活性,同时也没有使用\+\verb|priority|特性。\textsf{Chinese JFM}文件中同时增加了\+\verb|priority|的设置,配合\textsf{luatexja-adjust}包,可以进行有优先顺序的标点挤压。\label{sec:pr}

\section{使用方法}
\textsf{luatexja}在20200919版本后更新了JFM features特性,基本语法为
\begin{verbatim}
    jfm = <JFM name>[/{<JFM features>}]
\end{verbatim}
其中,\(\langle\)\textit{JFM features}\(\rangle\)可以是一个或多个由逗号分隔的键或键值对。

\textsf{Chinese JFM}提供了\textsf{zh\_CN}、\textsf{zh\_TW}与\textsf{ja\_JP}三个文件,分别适用于简中字体(标点偏靠)、繁中字体(标点中置)与日文字体(冒号、分号中置,其他偏靠);在3个JFM文件中,均提供了\+\verb|quanjiao|、\verb|banjiao|与\+\verb|kaiming|三个基本特性。

在使用\CTeX{}宏集或文档类时,只需要使用
\begin{verbatim}
    \documentclass[punct = zh_CN/quanjiao]{ctexart}
\end{verbatim}
或
\begin{verbatim}
    \usepackage[punct = zh_CN/kaiming]{ctex}
\end{verbatim}
即可。

在使用\textsf{luatexja}或\textsf{ltj}文档类时,需要在调用\textsf{luatexja-fontspec}宏包后,以\begin{verbatim}
    YokoFeatures = {JFM = {<JFM name>}}
\end{verbatim}
或
\begin{verbatim}
    TateFeatures = {JFM = {<JFM name>}}
\end{verbatim}
的形式调用。

横排时,使用
\begin{verbatim}
    YokoFeatures = {JFM = {zh_CN/quanjiao}}
\end{verbatim}
直排时,使用
\begin{verbatim}
    TateFeatures = {JFM = {zh_CN/{quanjiao,vert}}}
\end{verbatim}
{\color{red}注意:在直排时,\emph{必须}同时使用\+\verb|vert|特性。}

\section{特性一览}
除上文介绍的\+\verb|quanjiao|、\verb|banjiao|、\verb|kaiming|与\+\verb|vert|特性外,\textsf{Chinese JFM}还提供了以下特性。受OpenType特性影响,所有\textsf{Chinese JFM}特性均为4个小写字母。特性说明中,等号表明该特性可以带值,等号后面的数字是该特性的默认值。

\subsection{基本特性}
\begin{description}
    \item[\texttt{prop}] ``proportional'',「比例宽度」特性。适用于比例宽度的日文字体,比之于\textsf{lua\-texja}自带的\textsf{jfm-prop.lua}与\textsf{jfm-propv.lua},\verb|prop|特性将此两者整合的同时,也提供了对标点间距的调整功能。仅\textsf{ja\_JP}拥有。
    \item[\texttt{hwcl}] ``half-width colon'',「半宽冒号」特性。适用于简中直排时,冒号分号只占半宽的字体(如华文宋体等)。仅\textsf{zh\_CN}拥有。
    \item[\texttt{hwex}] ``half-width exclamation mark'',「半宽感叹号」特性。开启此特性,繁中、日文横排感叹号将视作半宽,而非全宽。\textsf{zh\_TW}、\textsf{ja\_JP}拥有。
    \item[\texttt{fzpr = 0.15}] ``\textit{Fangzheng} parenthesis'',「方正括号」特性。部分方正字体会将除引号外的括号做在靠中间位置。开启本特性,可以将括号调整至「正常」位置。仅\textsf{zh\_CN}拥有。
    \item[\texttt{hang = 0}] 「标点悬挂」特性。用于悬挂标点。将可悬挂标点分5个等级:\verb|1|逗号、句号;\verb|2|冒号、分号;\verb|4|右括号、右引号、分隔号;\verb|8|问号、叹号;\verb|16|左括号、左引号、在行首的分隔号。5种类型的标点以按位枚举的结果确定最终悬挂的标点种类:如\+\verb|11|表示悬挂逗号、句号,冒号、分号和问号、叹号;\verb|31|表示悬挂所有。
\end{description}

\subsection{特殊特性}
以下几个特性,可用于将某地区字体的标点符号移动至其他地区的习惯位置。特性的默认值是基于思源而设置的,如果使用其他字体,可能需要自行尝试调整。
\begin{description}
    \item[\texttt{tocn}] 用于将标点设为中国大陆习惯的偏靠位置,但是全宽问号不做处理。
    \item[\texttt{tojp}] 用于将标点设为日本习惯的偏靠位置,冒号分号仍然中置。
    \item[\texttt{totw}] 用于将标点设为台湾地区习惯的中置位置。
\end{description}

\section{挤压顺序}
如第\ref{sec:pr}节所述,\textsf{Chinese JFM}中配有\+\verb|priority|的设置,配合\textsf{luatexja-adjust}包,可以进行有优先顺序的标点挤压。\textsf{Chinese JFM}沿袭传统铅字排版与现代DTP软件(如InDesign等)的习惯,认为标点只占半宽,全宽标点是半宽标点加上半宽铅空的结果。所谓的“标点挤压”更应说成是“标点间距调整”,调整顺序如下。
\begin{itemize}
    \item 最先给句号、问号、叹号插空;
    \item 其次给顿号、逗号、冒号、分号插空;
    \item 最后调整引号、括号前后与间隔号两边的空格;
    \item 如果进行上述调整后,仍无法达到行长要求,最后才会进行字间字距调整。
\end{itemize}

根据标点位置,可以将标点分为偏靠标点、中置标点与全宽标点。对于偏靠标点,插空指在半宽标点后插入一个半宽铅空,对于中置标点,插空指在半宽标点前后各插入一个1/4宽铅空。全宽标点无空白可调整的,不作调整。

\section{将来计划}
\begin{enumerate}
    \item 添加直排时的行间标点设置。
\end{enumerate}

\nocite{*}
\begin{thebibliography}{9}
    \bibitem{LuaTeX-ja} Lua\TeX-jaプロジェクトチーム.\textit{Lua\TeX-jaパッケージ}[EB/OL].version 20210521.0,(2021-05-21)\\\mbox{}
    \hfill\url{https://ctan.org/pkg/luatexja}

    \bibitem{CTeX} CTEX.ORG.\textit{\CTeX{}宏集手册}[EB/OL].version 2.5.6,(2021-03-14)\\\mbox{}
    \hfill\url{https://ctan.org/pkg/ctex}

    \bibitem{孔雀计划} 刘庆(Eric Q L).\textit{孔雀计划:中文字体排印的思路}[EB/OL].\\\mbox{}
    \hfill\url{https://thetype.com/kongque/}
\end{thebibliography}

\end{document}
